%%%%%%%%%%%%%%%%%%%%%%%%%%%%%%%%%%%%%%%%%%%%%%%%%%%%%%%%%%%%%%%%%%% 
%                                                                 %
%                            CHAPTER ONE                          %
%                                                                 %
%%%%%%%%%%%%%%%%%%%%%%%%%%%%%%%%%%%%%%%%%%%%%%%%%%%%%%%%%%%%%%%%%%% 
 
\chapter{Introduction} \label{sec:introduction}
 
\section{Humpback Whales}

\section{Current Identification Methods}

Photo-identification of Humpback whale flukes thas been attempted since the early 90s \cite{mizroch1990computer} using computational aid.
While early efforts mostly relied on a manual description of the fluke that would then be matched, later efforts have involved matching flukes based on automated analysis of both the patterns on the ventral side of the fluke.
It is worth noting however that as shown in \cite{blackmer2000temporal}, the trailing edge changes less with age than surface patterns on the fluke, which means that it can (potentially) be a more reliable identifier over time.
That said, trailing edges are hard to photograph well, requiring high resolution imagery and a consistent angle between fluke and photographer.

\subsection{Based on Trailing Edge}

In \cite{mizroch1990computer}, information about the trailing edge is catalogued and 
While there is not a lot of literature on using trailing edge contours for identifying Humpback whales


\subsection{Based on general Fluke appearance}


\section{Hotspotter}

\section{The Dataset}

%%% Local Variables: 
%%% mode: latex
%%% TeX-master: t
%%% End: 
