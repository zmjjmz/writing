%%%%%%%%%%%%%%%%%%%%%%%%%%%%%%%%%%%%%%%%%%%%%%%%%%%%%%%%%%%%%%%%%%% 
%                                                                 %
%                            ABSTRACT                             %
%                                                                 %
%%%%%%%%%%%%%%%%%%%%%%%%%%%%%%%%%%%%%%%%%%%%%%%%%%%%%%%%%%%%%%%%%%% 
 
\specialhead{ABSTRACT}

Photographic identification of humpback whale (Megaptera novaeangliae) flukes (i.e.\ their tail) is an important task in marine ecology, and is used in tracking migration patterns and estimating populations \cite{blackmer2000temporal, calambokidis2008splash}.
In this thesis, we lay out a method that automates the photo-identification of humpback flukes using the ``trailing edge'' of the fluke.
The method uses convolutional networks to identify keypoints on the fluke and possible trailing edge locations.
It then uses this information to extract a detailed trailing edge and its curvature, which is then matched to other trailing edges in the database via dynamic time warping.
Using this method, we achieve nearly 80\% top-1 ranking accuracy on a large subset of the SPLASH \cite{calambokidis2008splash} dataset consisting of about 400 identified individuals.
We also show that in combination with Hotspotter \cite{crall_hotspotter_2013}, a general pattern based matching algorithm, we can achieve 93\% accuracy.

To our knowledge, this is the first method that can achieve this level of accuracy on humpback fluke identification without extensive manual effort at inference time.



%Humpback whales (Megaptera novaeangliae) are some of the most iconic cetaceans inhabiting our oceans, and have historically been at risk for extinction due to human activity --- primarily hunting \cite{breiwick1983simulated}. 
%While they are currently rated as 'Least Concern' \cite{reilly2008megaptera} internationally, they are still at risk from various commercial operations, and tracking their population is important. 
%Using photographic identification of the Humpback flukes is a common and reliable method for tracking these populations \cite{blackmer2000temporal}.
%These flukes are often patterned and scarred in unique ways, allowing conservationists to identify individuals \cite{katona1990population}.
%However, until recently, most automated identification methods still relied on signficant manual effort to describe and identify the fluke, severely limiting the amount of humpbacks that can be tracked.


