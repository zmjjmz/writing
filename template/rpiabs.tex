%%%%%%%%%%%%%%%%%%%%%%%%%%%%%%%%%%%%%%%%%%%%%%%%%%%%%%%%%%%%%%%%%%% 
%                                                                 %
%                            ABSTRACT                             %
%                                                                 %
%%%%%%%%%%%%%%%%%%%%%%%%%%%%%%%%%%%%%%%%%%%%%%%%%%%%%%%%%%%%%%%%%%% 
 
\specialhead{ABSTRACT}

Humpback whales (Megaptera novaeangliae) are an important part of our ocean's ecosystems [citation needed], and have historically been at risk for extinction [citation needed]. 
While they are currently rated as 'Least Concern' [citation needed], tracking their migration patterns is important for helping the (currently small) population grow [citation needed]. 
In order to discern these migration patterns, conservationists need to be able to track individual humpback whales [citation needed].
One of the easiest (and cheapest) ways to do this is to watch for their tails as they breach the water surface [citation needed], giving a clear view of what is known as a Humpback 'fluke'.
These are often patterned and scarred in unique ways, allowing conservationists to identify individuals [citation needed].
However, until recently, most automated identification methods still rely on signficant manual effort to describe and identify the fluke, severely limiting the amount of humpbacks that can be tracked [citation needed?].

This thesis lays out a method that automates the identification of Humpback flukes directly from still images thereof, using the 'trailing edge' of the fluke.
Using this method, we achieve a fairly high top-1 ranking accuracy on a large dataset (consisting of about 400 identified individuals).
We also show that this method significantly helps the accuracy of a pure appearance based method, Hotspotter [citation needed], giving 89\% top-1 accuracy.

To our knowledge, this is the first method that can achieve this level of accuracy on Humpback fluke identification without any manual effort at test-time.

TODO: Put in citations


