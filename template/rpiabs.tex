%%%%%%%%%%%%%%%%%%%%%%%%%%%%%%%%%%%%%%%%%%%%%%%%%%%%%%%%%%%%%%%%%%% 
%                                                                 %
%                            ABSTRACT                             %
%                                                                 %
%%%%%%%%%%%%%%%%%%%%%%%%%%%%%%%%%%%%%%%%%%%%%%%%%%%%%%%%%%%%%%%%%%% 
 
\specialhead{ABSTRACT}

Humpback whales (Megaptera novaeangliae) are some of the most iconic cetaceans inhabiting our oceans, and have historically been at risk for extinction due to human activity --- primarily hunting \cite{breiwick1983simulated}. 
While they are currently rated as 'Least Concern' \cite{reilly2008megaptera} internationally, they are still at risk from various commercial operations, and tracking their population is important. 
Using photographic identification of the Humpback flukes is a common and reliable method for tracking these populations \cite{blackmer2000temporal}.
These flukes are often patterned and scarred in unique ways, allowing conservationists to identify individuals \cite{katona1990population}.
However, until recently, most automated identification methods still relied on signficant manual effort to describe and identify the fluke, severely limiting the amount of humpbacks that can be tracked.

This thesis lays out a method that automates the identification of Humpback flukes directly from still images thereof, using the 'trailing edge' of the fluke.
Using this method, we achieve a fairly high top-1 ranking accuracy on a large dataset (consisting of about 400 identified individuals).
We also show that this method significantly helps the accuracy of a pure appearance based method, Hotspotter \cite{crall_hotspotter_2013}, giving 93\% top-1 accuracy.

To our knowledge, this is the first method that can achieve this level of accuracy on Humpback fluke identification without any manual effort at test-time.



