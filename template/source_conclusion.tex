
We have shown that a passive, appearance-based identification approach (photographic censusing) is capable of producing a population estimate and generate other useful statistics for the Nairobi National Park.  It should be noted that the enlistment of citizen scientists to help collect image data was critical to its success and the amount of coverage and volume of images collected would not have been possible without their help. The prototype IBEIS computer vision system offered a highly parallelized procedure for ingesting collected images and leveraged state-of-the-art algorithms to locate, identify, and classify images of animals.  In summary, the following improvements and additions have been made to IBEIS and its support software: 1.) new software to submit data to IBEIS, 2.) new web-based interfaces for turking detection regions, viewpoints, quality, and sex and age information, and 3.) species, viewpoint, and quality Deep Convolutional Neural Networks to classify annotations for increased identification performance in the IBEIS.  

The resulting analytics produced by IBEIS can be used by the Nairobi National Park to monitor the zebra and giraffe populations.  The analytics can also be used to actively monitor the population health by analyzing the demographics and, with continued collection of movement information, better organize the physical attributes of the park (e.g.\ salt licks, added damns, altering roads, etc.).  Using the population estimates, the conservationists in charge of the Nairobi National Park can begin to make data-driven decisions to better tend for the health of the ecosystem.  As a reminder, the ecological data generated by the GZGC provides and incomplete (therefore biased) and ever-changing picture of the Nairobi National Park.  The importance of continual monitoring of the park using the procedure and computer vision techniques described in this thesis will not only lead to more accurate ecological data, but also help mature the algorithms used and reduce the effects of any sampling bias.

Future work on analyzing the populations within the NNP should place focus on completing and augmenting the data collected during the GZGC.  The continuation of this passive monitoring is required to eventually produce a robust, unbiased population estimate, analyze seasonal trends in the ecosystem of the park, and potentially estimate the average life-expectancy of the animals.  The authors would also like to see additional studies performed at other conservancies across Kenya and in other countries where identifying native or endangered populations of animals is a main goal.  Future data collection events should put financial resources towards better solving the time and GPS synchronization by providing cameras to participants that sport in-camera GPS;  in-camera GPS significantly reduces the complexity of the collection and reduces the overall inaccuracy in the collection reconstruction process.  Finally, we would like a comparative study to be performed that would analyze which population counting methods are the most accurate or robust to the various real-world sampling errors encountered in the field.

The GZGC was a success in all three of its goals: to collect images using volunteer citizen scientists, offer engagement and a rewarding experience for the volunteers, and ultimately producing a population estimate for the NNP.  Potential uses for IBEIS include the continual monitoring of the NNP population and the monitoring of populations across Kenya; IBEIS and its computer vision algorithms will continue to mature and add new recognized species as the system is used for population monitoring.  In conclusion, as of March of 2015 there are 1,258 known individual plains zebras and 466 Masai giraffes in the Nairobi National park.  These known individuals are estimated to make up 47.1\% to 64.8\% of the actual population of plains zebras and 61.7\% to 100.0\% for Masai giraffes.
